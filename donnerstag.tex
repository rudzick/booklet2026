
% time: Thursday 09:00
% URL: https://pretalx.com/fossgis2023/talk/fossgis2026-84107-trailscan-qgis-plugin-zur-kartierung-forstlicher-ruckegassen-in-laserscanning-daten/

%
\newTimeslot{09:00}
\noindent\abstractHSzwei{%
  Tanja Kempen%
}{%
  TrailScan: QGIS-Plugin zur Kartierung forstlicher Rückegassen in Laserscanning Daten%
}{%
}{%
  Im Waldbodenschutz spielen Rückegassen eine wichtige Rolle, da sie die Befahrung mit schweren
  Maschinen auf festgelegte Bereiche reduzieren sollen. Mit dem neuen **QGIS-Plugin TrailScan**
  können diese Flächen im Wald automatisiert kartiert werden. Für die Anwendung werden **frei
  verfügbare Airborne Laserscanning Daten** verwendet. Das Plugin nutzt **PDAL** zur Prozessierung
  der 3D-Punktwolken. Mithilfe eines **CNN-Modells** werden dann Raster-Karten mit den Rückegassen
  ausgegeben.%
}%


%%%%%%%%%%%%%%%%%%%%%%%%%%%%%%%%%%%%%%%%%%%

% time: Thursday 09:05
% URL: https://pretalx.com/fossgis2023/talk/fossgis2026-84286-routing-to-go-openrouteservice-in-qfield/

%
\newTimeslot{09:05}
\noindent\abstractHSvier{%
  Till Frankenbach, Julian Psotta%
}{%
  Routing to Go: openrouteservice in QField%
}{%
}{%
  Routing direkt im Feld: ORStools und QField bringen openrouteservice auf mobile Geräte~-- ein
  Ausblick.%
}%


%%%%%%%%%%%%%%%%%%%%%%%%%%%%%%%%%%%%%%%%%%%

% time: Thursday 09:10
% URL: https://pretalx.com/fossgis2023/talk/fossgis2026-83601-qgis-plugins-unter-qgis4-qt6-lauffahig-machen/

%
\newTimeslot{09:10}
\noindent\abstractHSvier{%
  Johannes Kröger%
}{%
  QGIS-Plugins unter QGIS4/Qt6 lauffähig machen%
}{%
}{%
  QGIS 4, auf Basis von Version 6 des Qt-Frameworks, ist gerade frisch erschienen und (d)ein Plugin
  läuft noch nicht damit? Ich zeige eine kurze Übersicht zu den nötigen Schritten zur Portierung,
  sowie zu hilfreichen Tools und bekannten Fallstricken.%
}%


%%%%%%%%%%%%%%%%%%%%%%%%%%%%%%%%%%%%%%%%%%%

% time: Thursday 09:15
% URL: https://pretalx.com/fossgis2023/talk/fossgis2026-84019-qfield-plugin-entwicklung-lernen/

%
\newTimeslot{09:15}
\noindent\abstractHSvier{%
  Heather Hillers%
}{%
  QField Plugin Entwicklung lernen%
}{%
}{%
  Das QField Vegetation Plugin-Projekt bietet Codebeispiele und Dokumentation, die Sie durch die
  Entwicklung eines eigenen QField Plugins führen. Es demonstriert Feature-Selektion,
  Feature-Bearbeitung und mehrere nützlicher Widgets. Wir präsentieren das Plugin kurz und geben
  Orientierungshilfen für den Lernpfad der QField Plugin-Entwicklung.%
}%


%%%%%%%%%%%%%%%%%%%%%%%%%%%%%%%%%%%%%%%%%%%

% time: Thursday 09:35
% URL: https://pretalx.com/fossgis2023/talk/fossgis2026-82300-qgis-formulare-effizienzsteigerung-bei-der-erfassung-von-umweltinformationen-des-nok/

%
\newTimeslot{09:35}
\noindent\abstractHSeins{%
  Niny Zamora%
}{%
  QGIS-Formulare Effizienzsteigerung bei der Erfassung von Umweltinformationen des NOK%
}{%
}{%
  Die Benutzerfreundlichkeit der Datenerfassung in QGIS lässt sich durch den Einsatz von Formularen
  erheblich verbessern. In diesem Vortrag wird am Beispiel der Verwaltung der vielfältigen
  Umweltinformationen beim Ausbau des Nord-Ostsee-Kanals gezeigt, wie Formulare konfiguriert werden
  und durch Ausdrücke und Python-Skripte eine Erfassung standardisierter Informationen bequem
  umgesetzt werden kann. Die zu erfassenden Informationen werden durch ein selbst entwickeltes
  Plugin gefiltert und angezeigt%
}%


%%%%%%%%%%%%%%%%%%%%%%%%%%%%%%%%%%%%%%%%%%%

% time: Thursday 09:35
% URL: https://pretalx.com/fossgis2023/talk/fossgis2026-84044-lidar-fusion-fur-resiliente-walder-fossgis-zur-integration-von-mls-uls-und-als/

%

\noindent\abstractHSzwei{%
  Svenja Dobelmann%
}{%
  LiDAR-Fusion für resiliente Wälder: FOSSGIS zur Integration von MLS, ULS und ALS%
}{%
}{%
  LiDAR-Technologien verändern die Art und Weise, wie wir Wälder erfassen und verstehen. Im Rahmen
  des Projekts FoResLab werden Daten aus mobilen, UAV- und flugzeugbasierten Scans kombiniert, um
  Waldstrukturen präzise zu analysieren und Indikatoren der Waldresilienz abzuleiten. Mithilfe
  freier Software wie QGIS und PDAL, innovativer Online-Tools wie der RSDB sowie leistungsstarker
  HPC-Systeme entwickeln wir effiziente und transparente Workflows die in dem Beitrag gezeigt und
  diskutiert werden.%
}%


%%%%%%%%%%%%%%%%%%%%%%%%%%%%%%%%%%%%%%%%%%%

% time: Thursday 09:35
% URL: https://pretalx.com/fossgis2023/talk/fossgis2026-83562-serious-vectorlayering-goodbye-wms-getmap-und-getfeatureinfo/

%

\noindent\abstractHSdrei{%
  Thorsten Hell%
}{%
  Serious Vectorlayering: Goodbye WMS GetMap und GetFeatureInfo%
}{%
}{%
  Wir zeigen, wie man mit wenigen, gezielten Anpassungen beim Einsatz von Vectorlayern in der
  Webkartographie automatisch zu besseren Ergebnissen kommt. Dieses Prinzip, bekannt als "`Falling in
  the pit of success"', bedeutet, dass durch durchdachte Prinzipien und gutes Design verlässliche
  Resultate ohne zusätzlichen Aufwand entstehen. Beispiele aus dem Digitalen Zwilling Wuppertal
  zeigen, wie sich Performance, Darstellung und Interaktion spürbar verbessern.%
}%


%%%%%%%%%%%%%%%%%%%%%%%%%%%%%%%%%%%%%%%%%%%

% time: Thursday 09:35
% URL: https://pretalx.com/fossgis2023/talk/fossgis2026-82991-hyperspektrale-erdbeobachtung-in-der-praxis-die-enmap-box-in-qgis/

%

\noindent\abstractHSvier{%
  Benjamin Jakimow%
}{%
  Hyperspektrale Erdbeobachtung in der Praxis~-- Die EnMAP-Box in QGIS%
}{%
}{%
  Die EnMAP-Box ist ein QGIS Plugin zur Visualisierung, Analyse und Prozessierung von hyper- und
  multispektralen Erdbeobachtungsdaten. Unsere Demo zeigt, wie EnMAP-Daten bezogen und in der
  EnMAP-Box fachgerecht ausgewertet werden können.%
}%


%%%%%%%%%%%%%%%%%%%%%%%%%%%%%%%%%%%%%%%%%%%

% time: Thursday 10:10
% URL: https://pretalx.com/fossgis2023/talk/fossgis2026-84215-forestpulse-geodatendienste-fur-deutschlands-walder/

%
\newTimeslot{10:10}
\noindent\abstractHSzwei{%
  Sebastian Schnell%
}{%
  ForestPulse: Geodatendienste für Deutschlands Wälder%
}{%
}{%
  ForestPulse entwickelt jährlich aktualisierte Informationsprodukte zu Waldfläche, Baumarten,
  Vitalität und Struktur für die Wälder Deutschlands, die als freie Geodatenservices zur Verfügung
  gestellt werden.  Basierend auf Fernerkundungs- und Felddaten werden zeitlich und inhaltlich
  konsistente Informationen bereitgestellt. Durch die Verwendung eigener Trainingsdaten ist es
  möglich vortrainierte Modelle regional anzupassen.  Im Beitrag werden die Dienste vorgestellt und
  erste Ergebnisse gezeigt.%
}%


%%%%%%%%%%%%%%%%%%%%%%%%%%%%%%%%%%%%%%%%%%%

% time: Thursday 10:10
% URL: https://pretalx.com/fossgis2023/talk/fossgis2026-82604-von-sensor-bis-karte-erste-schritte-mit-qgis-und-sensorthings-api/

%

\noindent\abstractHSdrei{%
  Michael Stein%
}{%
  Von Sensor bis Karte: Erste Schritte mit QGIS und SensorThings API%
}{%
}{%
  Der Vortrag zeigt, wie sich Sensordaten über die OGC SensorThings API in QGIS visualisieren
  lassen. Nach einer kurzen Einführung in das Konzept von SensorThings werden schrittweise Karten
  erstellt~-- von der Übersicht aller Sensoren bis hin zur detaillierten Darstellung eines einzelnen
  Datenstroms mit Jahreswerten und Diagramm. Abschließend wird gezeigt, wie aktuelle Messwerte einer
  Sensorart auf einer Karte dargestellt werden können.%
}%


%%%%%%%%%%%%%%%%%%%%%%%%%%%%%%%%%%%%%%%%%%%

% time: Thursday 11:10
% URL: https://pretalx.com/fossgis2023/talk/fossgis2026-83665-groe-qgis-projekte-im-dauerbetrieb-erfahrungen-tools-tipps-herausforderungen/

%
\newTimeslot{11:10}
\noindent\abstractHSeins{%
  Gabi Gebhardt-Weidl%
}{%
  Große QGIS-Projekte im Dauerbetrieb~-- Erfahrungen, Tools, Tipps \& Herausforderungen%
}{%
}{%
  QGIS-Projekte im Dauerbetrieb einsetzen~-- das geht. Wir berichten aus der Praxis nach acht Jahren
  produktiver Nutzung bei den Stadtwerken München.
  Unser zentrales QGIS Projekt hat > 1.200 Layer, tausende Nutze\textbackslash1(\textbackslash2) und wird mit QGIS \& QGIS
  Server verwendet.
  Der Vortrag zeigt, wie große QGIS-Projekte erfolgreich betrieben \& weiterentwickelt werden können.
  Wir zeigen unsere Lösungen \& selbst entwickelten Tools, berichten über Herausforderungen und
  teilen unsere Erfahrungen mit der Community.%
}%


%%%%%%%%%%%%%%%%%%%%%%%%%%%%%%%%%%%%%%%%%%%

% time: Thursday 11:10
% URL: https://pretalx.com/fossgis2023/talk/fossgis2026-84125-kollaboratives-gis-mit-jupyter-notebooks-und-jupytergis/

%

\noindent\abstractHSzwei{%
  Pirmin Kalberer%
}{%
  Kollaboratives GIS mit Jupyter Notebooks und JupyterGIS%
}{%
}{%
  Jupyter Notebooks sind aus der wissenschaftlichen Datenauswertung nicht mehr wegzudenken. Die
  webbasierte, interaktive Umgebung zur Erstellung von Notebook-Dokumenten wird häufig auch für
  geografische Datenanalysen verwendet.
  Eine neue Applikation im Jupyter-Ökosystem ist JupyterGIS, welche kollaborative Bearbeitung von
  GIS-Daten in Echtzeit unterstützt und das Python-API von QGIS zur Kartendarstellung nutzt.%
}%


%%%%%%%%%%%%%%%%%%%%%%%%%%%%%%%%%%%%%%%%%%%

% time: Thursday 11:10
% URL: https://pretalx.com/fossgis2023/talk/fossgis2026-83875-von-mapillary-zu-maproulette-radweglucken-automatisch-erkennen-und-erganzen/

%

\noindent\abstractHSvier{%
  Simon Metzler%
}{%
  Von Mapillary zu MapRoulette: Radweglücken automatisch erkennen und ergänzen%
}{%
}{%
  In diesem Lightning-Talk wird gezeigt, wie automatisch erkannte Verkehrszeichen aus Mapillary
  genutzt werden, um Lücken im Radwegenetz in OpenStreetMap zu finden und in einer
  MapRoulette-Challenge gezielt zu bearbeiten. Die Kampagne hat bereits über 200 km bisher nicht
  erfasste Radinfrastruktur in Deutschland ergänzt und lässt sich auch auf weitere Themen wie Tempo
  30 oder Zebrastreifen übertragen.%
}%


%%%%%%%%%%%%%%%%%%%%%%%%%%%%%%%%%%%%%%%%%%%

% time: Thursday 11:10
% URL: https://pretalx.com/fossgis2023/talk/fossgis2026-84270-osgeo-deegree-anwendertreffen/

%

\noindent\abstractOther{%
  Torsten Friebe, Stephan Reichhelm, Dirk Stenger%
}{%
  OSGeo deegree~-- Anwendertreffen%
}{%
}{%
  Zum Anwendertreffen sind Anwender:innen und Entwickler:innen herzlich eingeladen, die
  Netzwerkdienste wie WMS, WFS oder OGC API~-- Features mit dem OSGeo-Projekt deegree umsetzen oder
  dieses für die Zukunft planen. Wie jedes Jahr wird es als Einstieg einen kurzen Überblick über die
  aktuellen Änderungen im Projekt geben.%
}%
{%
  BoF1 (ZHG 001)%
}%



%%%%%%%%%%%%%%%%%%%%%%%%%%%%%%%%%%%%%%%%%%%

% time: Thursday 11:10
% URL: https://pretalx.com/fossgis2023/talk/fossgis2026-84290-neues-von-der-gbd-websuite/

%

\noindent\abstractOther{%
  Otto Dassau%
}{%
  Neues von der GBD WebSuite%
}{%
}{%
  Im Oktober 2025 wurde die Version 8.2 der GBD WebSuite (https://gbd-websuite.de) veröffentlicht.
  Sie bringt Verbesserungen beim Support von Rasterdaten, OGC Diensten und Interoperabilität, bei
  der ALKIS Suche sowie dem Drucken von Karten mit. Zudem wurden zahlreiche Optimierungen für die
  Administration und das Backend realisiert sowie umfangreiche Vorbereitungen für Innovationen in
  zukünftigen Versionen getroffen. Dieses und noch mehr möchten wir euch vorstellen und gemeinsam
  diskutieren.%
}%
{%
  BoF2 (ZHG 005)%
}%



%%%%%%%%%%%%%%%%%%%%%%%%%%%%%%%%%%%%%%%%%%%

% time: Thursday 11:10
% URL: https://pretalx.com/fossgis2023/talk/fossgis2026-83203-automatische-daten-und-stilgesteuerte-kartenoptimierung/

%

\noindent\abstractAnwBoFdrei{%
  Frank Elsinga%
}{%
  Automatische daten- und stilgesteuerte Kartenoptimierung%
}{%
}{%
  Der Vortrag zeigt, wie sich Kartenstile und zugrunde liegende Daten automatisch optimieren lassen,
  um Performance und Ressourcennutzung zu verbessern. Anhand praktischer Beispiele wird erläutert,
  wie daten- und stilgesteuerte Ansätze Ladezeiten verkürzen, Tile-Größen reduzieren und das
  Rendering effizienter machen.
  Takeaways
  - Welche Ansätze zur Optimierung von Kartenstilen/-daten existieren?
  - Warum diese Optimierungen die Performance einer Karte deutlich verbessern können%
}%


%%%%%%%%%%%%%%%%%%%%%%%%%%%%%%%%%%%%%%%%%%%

% time: Thursday 11:15
% URL: https://pretalx.com/fossgis2023/talk/fossgis2026-82557-neue-werkzeuge-furs-3d-mapping/

%
\newTimeslot{11:15}
\noindent\abstractHSvier{%
  Tobias Knerr%
}{%
  Neue Werkzeuge fürs 3D-Mapping%
}{%
}{%
  Die dreidimensionale Erfassung von Gebäuden und anderen Objekten in OpenStreetMap ist eine der
  komplexeren Aufgaben für Mapper. Neue Werkzeuge machen das 3D-Mapping zugänglicher und
  effizienter.%
}%


%%%%%%%%%%%%%%%%%%%%%%%%%%%%%%%%%%%%%%%%%%%

% time: Thursday 11:20
% URL: https://pretalx.com/fossgis2023/talk/fossgis2026-83575-uber-1000-apps-in-5-minuten-der-osm-apps-catalog/

%
\newTimeslot{11:20}
\noindent\abstractHSvier{%
  Christopher Lorenz%
}{%
  Über 1000 Apps in 5 Minuten~-- Der OSM Apps Catalog%
}{%
}{%
  Wer kennt es nicht: Da gab es doch diese eine Webanwendung mit der man einfach Informationen zu
  OpenStreetMap zu spezifischen Themen hinzufügen kann. Im OSM Apps Catalog findet man viel wenn
  nicht sogar alle öffentliche Apps, Webseiten und Karten rund um OpenStreetMap.%
}%


%%%%%%%%%%%%%%%%%%%%%%%%%%%%%%%%%%%%%%%%%%%

% time: Thursday 11:30
% URL: https://pretalx.com/fossgis2023/talk/fossgis2026-84317-winding-paths-on-deep-maps/

%
\newTimeslot{11:30}
\noindent\abstractAnwBoFdrei{%
  Anastasia Bauch%
}{%
  Winding Paths on Deep Maps%
}{%
}{%
  Deep Maps sollen ein Gegenentwurf zu als "`autoritär"' kritisierten thematischen Karten mit einem
  Einzelautor sein. Stattdessen können in Deep Maps verschiedene Perspektiven dargestellt werden, um
  diese Multiperspektivität zu Navigieren präsentiert das vorgestellte Projekt einen Ansatz eigene
  Pfade über diese Karten zu generieren und so Interaktion mit dem eigenen Pfad räumlich nahen
  Perspektiven anzuregen.%
}%


%%%%%%%%%%%%%%%%%%%%%%%%%%%%%%%%%%%%%%%%%%%

% time: Thursday 11:45
% URL: https://pretalx.com/fossgis2023/talk/fossgis2026-83063-einfuhrung-von-qgis-als-standard-desktop-gis-arbeitsplatz-der-bayerischen-umweltverwa/

%
\newTimeslot{11:45}
\noindent\abstractHSeins{%
  Carolin von Groote-Bidlingmaier, Christian Strobl%
}{%
  Einführung von QGIS als Standard-Desktop-GIS-Arbeitsplatz der Bayerischen Umweltverwa%
}{%
}{%
  Geographische Informationssysteme (GIS) bilden eine wichtige Arbeitsgrundlage für die
  Geodatenverarbeitung der Bayerischen Umweltverwaltung. Die bisher im Einsatz befindlichen GIS
  sollen in weiten Bereichen durch QGIS abgelöst werden. Für die Inbetriebnahme sind einige Aspekte
  hinsichtlich IT-Sicherheit, Performance, Interoperabilität und Betrieb geklärt worden. Diese
  sollen im Rahmen des Vortrags näher vorgestellt werden, um anderen Institutionen einen
  vergleichbaren Prozess zu erleichtern.%
}%


%%%%%%%%%%%%%%%%%%%%%%%%%%%%%%%%%%%%%%%%%%%

% time: Thursday 11:45
% URL: https://pretalx.com/fossgis2023/talk/fossgis2026-84119-diskrete-globale-gittersysteme-fur-die-raum-zeitliche-aggregation-und-visualisierung/

%

\noindent\abstractHSzwei{%
  Michael Scholz%
}{%
  Diskrete globale Gittersysteme für die raum-zeitliche Aggregation und Visualisierung%
}{%
}{%
  Die raum-zeitliche Aggregation großer (Punkt-)Datenmengen für die kartografische
  Web-Visualisierung stellt eine Herausforderung dar, die auf unterschiedliche Weisen angegangen
  werden kann. Anhand zweier Praxisbeispiele~-- auf Datenbankebene in PostgreSQL und im
  Streamprozessor-Framework Apache Flink~-- wird die Verwendung von Ubers H3-Programmbibliothek für
  die hexagonale Aggregation in einem Discrete Global Grid System (DGGS) vorgestellt und diskutiert.%
}%


%%%%%%%%%%%%%%%%%%%%%%%%%%%%%%%%%%%%%%%%%%%

% time: Thursday 11:45
% URL: https://pretalx.com/fossgis2023/talk/fossgis2026-84320-open-source-dashboard-zur-visualisierung-von-gemeinschaftlich-erfassten-gebaude-daten/

%

\noindent\abstractHSdrei{%
  Theodor Rieche%
}{%
  Open Source Dashboard zur Visualisierung von gemeinschaftlich erfassten Gebäude-Daten%
}{%
}{%
  Der Vortrag stellt die Entwicklung eines interaktiven Grafana-Dashboards vor, um die
  raum-zeitlichen Erfassungsaktivitäten von Gebäudemerkmalen (wie Nutzung, Alter oder Baumaterial)
  des Citizen Science Projektes "`Colouring Dresden"' erlebbar zu machen. So soll die Sichtbarkeit für
  das Projekt erhöht werden, und auch Motivation für neue Erfassungen gegeben werden. Der Vortrag
  geht auf die Erfahrungen aus dem Projekt ein und stellt die verwendeten technischen Komponenten
  und deren Entwicklung vor.%
}%


%%%%%%%%%%%%%%%%%%%%%%%%%%%%%%%%%%%%%%%%%%%

% time: Thursday 11:45
% URL: https://pretalx.com/fossgis2023/talk/fossgis2026-83981-aufbau-und-aktualisierung-einer-osm-basierten-karten-mit-osm2pgsql/

%

\noindent\abstractHSvier{%
  Mathias Gröbe%
}{%
  Aufbau und Aktualisierung einer OSM-basierten Karten mit osm2pgsql%
}{%
}{%
  Die Software osm2pgsql ermöglicht einen konfigurierbaren Import und eine Aktualisierung der
  OpenStreetMap-Daten in eine PostgreSQL/PostGIS-Datenbank. Hier dient die Erstellung und das
  Updaten der Karte, welche mit QGIS erstellt wird, als Beispiel. Die Infrastruktur von
  OpenStreetMap in Kombination mit der Software osm2pgsql macht eine fortlaufende automatische
  Aktualisierung möglich. In der Demosession werden die Konfigurations- und
  Kombinationsmöglichkeiten aufgezeigt.%
}%


%%%%%%%%%%%%%%%%%%%%%%%%%%%%%%%%%%%%%%%%%%%

% time: Thursday 12:20
% URL: https://pretalx.com/fossgis2023/talk/fossgis2026-84266-python-in-qgis-ein-blick-auf-die-schnittstellen-und-ihre-sicherheit/

%
\newTimeslot{12:20}
\noindent\abstractHSeins{%
  Isabelle Korsch, Johannes Kröger%
}{%
  Python in QGIS~-- ein Blick auf die Schnittstellen und ihre Sicherheit%
}{%
}{%
  Wie funktionieren Profile und Projekte in QGIS und welche Möglichkeiten bietet QGIS hier mit
  Python den Funktionsumfang zu erweitern und anzupassen? Zum Beispiel können in Profilen Plugins
  oder eigene Ausdrucksfunktionen liegen, in Projekten Makros und in Stilen Aktionen enthalten sein.
  Wir wollen das Potential dieser Schnittstellen betrachten und werfen dabei auch einen kritischen
  Blick auf sicherheitsrelevante Aspekte.%
}%


%%%%%%%%%%%%%%%%%%%%%%%%%%%%%%%%%%%%%%%%%%%

% time: Thursday 14:15
% URL: https://pretalx.com/fossgis2023/talk/fossgis2026-84186-open-source-tools-zur-erstellung-von-3d-tiles-erfahrungen-und-herausforderungen/

%
\newTimeslot{14:15}
\noindent\abstractHSdrei{%
  Martin Alzueta%
}{%
  Open Source Tools zur Erstellung von 3D Tiles~-- Erfahrungen und Herausforderungen%
}{%
}{%
  Ein praxisnaher Einblick in die Erstellung von 3D Tiles für einen digitalen Zwilling mit Open
  Source Tools~-- von der Datenaufbereitung verschiedener Geodatenformate bis zu Herausforderungen
  bei Konvertierung und Visualisierung.%
}%


%%%%%%%%%%%%%%%%%%%%%%%%%%%%%%%%%%%%%%%%%%%

% time: Thursday 14:15
% URL: https://pretalx.com/fossgis2023/talk/fossgis2026-82998-mapbender-anwendertreffen/

%

\noindent\abstractOther{%
  Genoveva Pottgiesser%
}{%
  Mapbender Anwendertreffen%
}{%
}{%
  Mapbender (https://mapbender.org/) ist ein Open-Source-WebGIS, das bereits seit über 20 Jahren
  entwickelt und vielfältig eingesetzt wird. Die Software zur Erstellung von webbasierten
  Kartenanwendungen bietet mächtige Werkzeuge zur Erfassung, Anzeige, Bearbeitung und Verwaltung von
  Geodaten.
  Im Rahmen der FOSSGIS möchten wir in einem Mapbender Anwendertreffen alle Interessierten zu
  Austausch, Diskussion und Networking einladen.%
}%
{%
  BoF1 (ZHG 001)%
}%



%%%%%%%%%%%%%%%%%%%%%%%%%%%%%%%%%%%%%%%%%%%

% time: Thursday 14:15
% URL: https://pretalx.com/fossgis2023/talk/fossgis2026-82548-indoor-osm/

%

\noindent\abstractOther{%
  Tobias Knerr, Volker Krause%
}{%
  Indoor OSM%
}{%
}{%
  Ein Treffen für alle, die als Mapper oder Entwickler mit Indoor-Karten in OpenStreetMap zu tun
  haben%
}%
{%
  BoF2 (ZHG 005)%
}%



%%%%%%%%%%%%%%%%%%%%%%%%%%%%%%%%%%%%%%%%%%%

% time: Thursday 14:15
% URL: https://pretalx.com/fossgis2023/talk/fossgis2026-84073-openstreetmap-fur-und-mit-offentlicher-verwaltung-und-unternehmen/

%

\noindent\abstractAnwBoFdrei{%
  Tobias Jordans, Lars Lingner, Christopher Lorenz%
}{%
  OpenStreetMap für und mit öffentlicher Verwaltung und Unternehmen%
}{%
}{%
  Wie können Organisationen der öffentlichen Hand und Unternehmen OpenStreetMap-Daten sinnvoll
  nutzen~-- und gleichzeitig dazu beitragen, OSM weiter zu verbessern? Welche Best Practices haben
  sich etabliert, und welche typischen Stolpersteine gilt es zu vermeiden?%
}%


%%%%%%%%%%%%%%%%%%%%%%%%%%%%%%%%%%%%%%%%%%%

% time: Thursday 14:50
% URL: https://pretalx.com/fossgis2023/talk/fossgis2026-84253-die-mobidata-bw-integrationsplattform-public-money-code-und-data/

%
\newTimeslot{14:50}
\noindent\abstractHSeins{%
  Thorsten Fröhlinghaus%
}{%
  Die MobiData BW® Integrationsplattform: Public Money, Code und Data%
}{%
}{%
  Die MobiData BW® Integrationsplattform ist ein von der Nahverkehrsgesellschaft Baden-Württemberg
  und Dienstleistern entwickeltes Open-Source-Projekt. Kommunale und kommerzielle Mobilitätsdaten zu
  u. a. Sharing-Angeboten, Parken, Ladesäulen und Baustellen werden rechtlich und technisch
  gebündelt und über offene APIs bereitgestellt. MobiData BW® sichert dabei Qualität und Aktualität
  und erfüllt durch die Veröffentlichung auf der Mobilithek kommunale Datenlieferungspflichten.%
}%


%%%%%%%%%%%%%%%%%%%%%%%%%%%%%%%%%%%%%%%%%%%

% time: Thursday 14:50
% URL: https://pretalx.com/fossgis2023/talk/fossgis2026-84333-actinia-copilot-urban-effizientere-stadtanalysen-mit-ki-und-open-source/

%

\noindent\abstractHSzwei{%
  Markus Neteler, Hinrich Paulsen%
}{%
  actinia-copilot URBAN: Effizientere Stadtanalysen mit KI und Open Source%
}{%
}{%
  "`actinia-Copilot URBAN"'~-- ein von der European Space Agency (ESA) kofinanziertes
  Open-Source-KI-System~-- verbindet Large Language Models (LLM) mit Geodatenverarbeitung. Es
  ermöglicht Stadtplanern und Kommunen, Klimaanpassungsprojekte durch natürliche Sprachabfragen
  umzusetzen, indem komplexe Geo- und Erdbeobachtungsdaten ohne tiefgreifende technische
  Fachkenntnisse zugänglich gemacht werden.%
}%


%%%%%%%%%%%%%%%%%%%%%%%%%%%%%%%%%%%%%%%%%%%

% time: Thursday 14:50
% URL: https://pretalx.com/fossgis2023/talk/fossgis2026-84268-moderne-cloud-daten-in-qgis/

%

\noindent\abstractHSvier{%
  Isabelle Korsch, Johannes Kröger%
}{%
  Moderne Cloud-Daten in QGIS%
}{%
}{%
  Cloud-optimized GeoTIFF, COPC, FlatGeoBuf, PMTiles, GeoParquet...~-- Moderne "`Cloud"-Formate haben
  in den letzten Jahren Einzug in die GIS-Welt gehalten. Für manche bereits ein alter Hut, aber
  nicht jed\textbackslash1(\textbackslash2) hat sich schon einmal rangetraut. In dieser Demosession zeigen wir in praktischen
  Beispielen wie der Umgang mit cloud-optimierten Datensätzen in QGIS gelingt und was ihre Vorteile
  sind.%
}%


%%%%%%%%%%%%%%%%%%%%%%%%%%%%%%%%%%%%%%%%%%%

% time: Thursday 15:25
% URL: https://pretalx.com/fossgis2023/talk/fossgis2026-82394-ml-mit-satellitenbildern-in-der-geo-engine-eine-operationalisierung-von-ml-diensten/

%
\newTimeslot{15:25}
\noindent\abstractHSzwei{%
  Dr. Christian Beilschmidt%
}{%
  ML mit Satellitenbildern in der Geo Engine: Eine Operationalisierung von ML-Diensten%
}{%
}{%
  Die Geo Engine ermöglicht Geoanalysen in der Cloud, von der Definition von Workflows über
  OGC-Schnittstellen bis zu Python-Notebooks. Der Vortrag zeigt die Erstellung einer
  Machine-Learning-Anwendung mit Sentinel-2-Bildern und einem ML-Klassifikator in Geo Engine. Es
  wird demonstriert, wie Modelle ins ONNX-Format transformiert, registriert und in Workflows
  integriert werden, sowie die Operationalisierung bestehender ML-Modelle wie eine Wolkenmaskierung.%
}%


%%%%%%%%%%%%%%%%%%%%%%%%%%%%%%%%%%%%%%%%%%%

% time: Thursday 15:25
% URL: https://pretalx.com/fossgis2023/talk/fossgis2026-83718-postgis-stletters-worte-sagen-manchmal-mehr/

%

\noindent\abstractHSdrei{%
  Astrid Emde%
}{%
  PostGIS ST\_Letters~-- Worte sagen manchmal mehr%
}{%
}{%
  Warum nach Geodaten suchen, wo es doch die PostGIS-Funktion ST\_Letters gibt, die im Nu Geometrien
  erzeugt?
  ST\_Letters ist eine recht unbekannte Funktion, die einen Text übergeben bekommt und diesen als
  Geometrie ausgibt. Genauer wird die Zeichenfolge von ST\_Letters als Multipolygon ausgegeben.
  https://postgis.net/docs/ST\_Letters.html%
}%


%%%%%%%%%%%%%%%%%%%%%%%%%%%%%%%%%%%%%%%%%%%

% time: Thursday 15:30
% URL: https://pretalx.com/fossgis2023/talk/fossgis2026-84105-automatische-prompt-optimierung-mit-dspy/

%
\newTimeslot{15:30}
\noindent\abstractHSdrei{%
  Maximilian Herbers%
}{%
  Automatische Prompt‑Optimierung mit DSPy%
}{%
}{%
  Der Aufstieg von LLMs und VLMs eröffnet neue Wege der Datenverarbeitung und -analyse, wobei die
  Qualität der Ergebnisse stark von der Formulierung der Prompts abhängt. In diesem Lightning Talk
  wird das Open-Source-Framework DSPy vorgestellt, das die Entwicklung, das Testen und die
  automatisierte Optimierung von Prompts in einen reproduzierbaren, recheneffizienten
  Python-Workflow integriert.%
}%


%%%%%%%%%%%%%%%%%%%%%%%%%%%%%%%%%%%%%%%%%%%

% time: Thursday 16:45
% URL: https://pretalx.com/fossgis2023/talk/fossgis2026-84334-qwc2-basierte-unterstutzung-der-manahmenplanung-im-nationalpark-schwarzwald/

%
\newTimeslot{16:45}
\noindent\abstractHSeins{%
  Christoph Dreiser%
}{%
  QWC2-basierte Unterstützung der Maßnahmenplanung im Nationalpark Schwarzwald%
}{%
}{%
  Im Referat für Geodatenmanagement der Nationalparkverwaltung wurde ein integriertes
  WebGIS-basiertes Managementinstrument erstellt. Es dient der räumlichen und zeitlichen
  Koordination von Maßnahmen auf der Fläche des Nationalparks, sowie der Sicherstellung ihrer natur-
  und artenschutzkonformen Umsetzung. Die Datenhaltung ist in PostGIS, die Dateneingabe und
  Abstimmungskommunikation zwischen Planern und naturschutzrechlichem Kontrollteam erfolgt im QGIS
  Web-Client (QWC2) mit dem Editor-Plugin.%
}%


%%%%%%%%%%%%%%%%%%%%%%%%%%%%%%%%%%%%%%%%%%%

% time: Thursday 16:45
% URL: https://pretalx.com/fossgis2023/talk/fossgis2026-84255-vom-desktop-ins-web-der-neue-atlas-der-schweiz/

%

\noindent\abstractHSzwei{%
  Alexander Müdespacher%
}{%
  Vom Desktop ins Web~-- der neue Atlas der Schweiz%
}{%
}{%
  Der Atlas der Schweiz ist der offizielle Nationalatlas und wird seit 1961 von der ETH Zürich
  herausgegeben. Die Desktop-Version (gestartet in 2016) basiert auf einem interaktiven 3D-Globus
  mit über 400 thematischen Karten. Dank neuer Webtechnologien~-- vom Rendering, über UI bis zu
  Geodatenformaten~-- konnte der Atlas ins Web migriert und einer grösseren Nutzerschaft zugänglich
  gemacht werden. Der Vortrag zeigt Konzeption, UX-Design, Architektur und den Einsatz offener
  Webtechnologien.%
}%


%%%%%%%%%%%%%%%%%%%%%%%%%%%%%%%%%%%%%%%%%%%

% time: Thursday 16:45
% URL: https://pretalx.com/fossgis2023/talk/fossgis2026-84209-optimierung-von-rasterdaten-mit-gdal/

%

\noindent\abstractHSdrei{%
  Dennis Davidsohn%
}{%
  Optimierung von Rasterdaten mit GDAL%
}{%
}{%
  Luftbilder spielen in vielen Anwendungen eine zentrale Rolle und werden meist als GeoTIFF
  gespeichert. Die Eigenschaften solcher Rasterdaten stellen jedoch häufig eine Herausforderung für
  Speicherung, Verarbeitung und Darstellung dar. In diesem Vortrag werden Möglichkeiten aufgezeigt,
  Rasterdaten mit GDAL zu optimieren.%
}%


%%%%%%%%%%%%%%%%%%%%%%%%%%%%%%%%%%%%%%%%%%%

% time: Thursday 16:45
% URL: https://pretalx.com/fossgis2023/talk/fossgis2026-83057-umstieg-auf-qgis-in-der-offentlichen-verwaltung/

%

\noindent\abstractOther{%
  ChristianSebaly%
}{%
  Umstieg auf QGIS in der öffentlichen Verwaltung%
}{%
}{%
  In der öffentlichen Verwaltung etabliert sich QGIS zunehmend als Standard im Desktop-GIS-Bereich.
  Neben der unabhängigen Nutzung ohne Herstellerbindung und einer breiten Unterstützung offener
  Standards sind kontinuierliche Weiterentwicklung, hohe Qualität, langfristige Zukunftssicherheit
  und digitale Souveränität entscheidende Faktoren, die für den Einsatz von QGIS sprechen.%
}%
{%
  BoF1 (ZHG 001)%
}%



%%%%%%%%%%%%%%%%%%%%%%%%%%%%%%%%%%%%%%%%%%%

% time: Thursday 16:45
% URL: https://pretalx.com/fossgis2023/talk/fossgis2026-83033-mapserver-anwendertreffen/

%

\noindent\abstractOther{%
  Henning Jensen%
}{%
  MapServer Anwendertreffen%
}{%
}{%
  MapServer ist eine Open-Source-Anwendung, mit der räumliche Daten über standardisierte
  OGC-Schnittstellen veröffentlicht werden können. Dieses Anwendertreffen bietet die Gelegenheit,
  MapServer-Nutzende (und solche, die es werden wollen) in einem konstruktiven Austausch
  zusammenzubringen.%
}%
{%
  BoF2 (ZHG 005)%
}%



%%%%%%%%%%%%%%%%%%%%%%%%%%%%%%%%%%%%%%%%%%%

% time: Thursday 16:45
% URL: https://pretalx.com/fossgis2023/talk/fossgis2026-84059-lizmap-webclient-mit-qgis-server-und-py-qgis-server/

%

\noindent\abstractAnwBoFdrei{%
  Günter Wagner%
}{%
  Lizmap Webclient mit QGIS-Server und Py-QGIS-Server%
}{%
}{%
  Diese Fragestunde, kombiniert mit Demo-Beispielen und ggf. Erfahrungsberichten, ermöglicht
  Interessierten einen Einblick in den Webclient Lizmap in Kombination mit dem QGIS-Server (und dem
  Py-QGIS-Server). Neben den Fragen der Teilnehmer werden spezielle Funktionen und Neuerungen
  vorgestellt. Dabei ist die Expert:innenfragestunde sowohl für neu Interessierte als auch für
  Anwender vom Lizmap Webclient interessant.%
}%


%%%%%%%%%%%%%%%%%%%%%%%%%%%%%%%%%%%%%%%%%%%

% time: Thursday 16:50
% URL: https://pretalx.com/fossgis2023/talk/fossgis2026-84022-indikatorenberechnung-mit-pygeoapi-fur-ein-sozialraummonitoring-in-kommonitor/

%
\newTimeslot{16:50}
\noindent\abstractHSvier{%
  Isabell Zerria, Sebastian Drost%
}{%
  Indikatorenberechnung mit pygeoapi für ein Sozialraummonitoring in KomMonitor%
}{%
}{%
  Die Open Source Software KomMonitor ermöglicht die Berechnung komplexer Leitindikatoren auf Basis
  von sozio-demografischen Geo- und Zeitreihendaten. Zu diesem Zweck wurde pygeoapi und prefect als
  Prozessmanager in die Softwarearchitektur von KomMonitor integriert, um fachspezifische Prozesse
  über den OGC API~-- Processes Schnittstellenstandard bereitzustellen. Dieser Ansatz ermöglicht die
  Ableitung komplexer Sozialindikatoren zur Unterstützung eines umfassenden Sozialraummonitorings.%
}%


%%%%%%%%%%%%%%%%%%%%%%%%%%%%%%%%%%%%%%%%%%%

% time: Thursday 17:20
% URL: https://pretalx.com/fossgis2023/talk/fossgis2026-84168-management-von-kartenstilen-mit-ogc-api-styles/

%
\newTimeslot{17:20}
\noindent\abstractHSzwei{%
  Jan Suleiman%
}{%
  Management von Kartenstilen mit OGC API~-- Styles%
}{%
}{%
  Der Standard OGC API~-- Styles ermöglicht die einheitliche Verwaltung und Veröffentlichung von
  Kartenstilen über offene Schnittstellen. Der Vortrag erläutert Konzepte, Vorteile und bestehende
  Implementierungen wie GeoServer und GeoStyler. Zudem werden aktuelle Lücken und künftige
  Entwicklungen aufgezeigt, um Stilmanagement in Geodateninfrastrukturen interoperabler und
  effizienter zu gestalten.%
}%


%%%%%%%%%%%%%%%%%%%%%%%%%%%%%%%%%%%%%%%%%%%

% time: Thursday 17:20
% URL: https://pretalx.com/fossgis2023/talk/fossgis2026-83731-sentinel-analysis-ready-data-freie-datenprodukte-und-tools/

%

\noindent\abstractHSdrei{%
  Guido Riembauer%
}{%
  Sentinel Analysis Ready Data~-- freie Datenprodukte und Tools%
}{%
}{%
  Sentinel-1 und Sentinel-2 liefern reichhaltige Erdbeobachtungsdaten, deren Vorverarbeitung jedoch
  komplex ist. Der Vortrag zeigt praxisnah, wie frei verfügbare Analysis Ready Data-Produkte und
  Open-Source-Tools (u.a. FORCE, SADASADAM, GRASS-GIS) den Einstieg erleichtern, Zeitreihenanalysen
  ermöglichen und Sentinel-Daten für GIS-Workflows direkt nutzbar machen.%
}%


%%%%%%%%%%%%%%%%%%%%%%%%%%%%%%%%%%%%%%%%%%%

% time: Thursday 17:55
% URL: https://pretalx.com/fossgis2023/talk/fossgis2026-84295-unsere-daten-neu-denken-twin-information-system-twis/

%
\newTimeslot{17:55}
\noindent\abstractHSeins{%
  Dr. Heino Rudolf%
}{%
  Unsere Daten neu denken~-- Twin Information System (TwIS)%
}{%
}{%
  Viele Städte erstellen sich Digitale Zwillinge, dabei geht es zumeist um hochaufgelöste
  3D-Stadtmodelle. (Ein nutzbarer Digitaler Zwilling des Autos ist aber nicht das 3D-Abbild.)
  Ein nutzbarer Urbaner Digitaler Zwilling ist ein dynamischer digitaler Repräsentant der Stadt, um
  den aktuellen und zukünftigen Zustand der Stadt zu erkennen und Einfluss auf Prozesse und die
  weitere Stadtentwicklung zu nehmen.
  Mit GIS ist das nicht umzusetzen. Deshalb haben wir eine neue Technologie kreiert: TwIS.%
}%


%%%%%%%%%%%%%%%%%%%%%%%%%%%%%%%%%%%%%%%%%%%

% time: Thursday 17:55
% URL: https://pretalx.com/fossgis2023/talk/fossgis2026-83049-apache-superset-und-ogc-wfs-offene-standards-fur-business-intelligence/

%

\noindent\abstractHSzwei{%
  Charlie Liebscher, Jan Suleiman%
}{%
  Apache Superset und OGC WFS~-- Offene Standards für Business Intelligence%
}{%
}{%
  Apache Superset ist eine Open-Source Business Intelligence Plattform und Bestandteil des
  CivitasConnect UDP Core. Im Rahmen des Förderprojekts Connected Urban Twins (BMWSB, KTS) wurde die
  Implementierung von OGC WFS als Datenquelle beauftragt und entwickelt. Damit wird die
  Geodatenkomponente von Apache Superset im Blick auf offene Datenstandards gestärkt.%
}%


%%%%%%%%%%%%%%%%%%%%%%%%%%%%%%%%%%%%%%%%%%%

% time: Thursday 17:55
% URL: https://pretalx.com/fossgis2023/talk/fossgis2026-84332-geospatial-narratives-kritische-lesbarkeit-von-satellitenbildern/

%

\noindent\abstractHSdrei{%
  Julian Peschel%
}{%
  Geospatial Narratives~-- kritische Lesbarkeit von Satellitenbildern%
}{%
}{%
  Wie können Satellitendaten als narratives und analytisches Medium im Journalismus
  kontextualisiert, überprüfbar und zitierfähig gemacht werden? Der Vortrag stellt die Plattform
  [Geospatial Narratives](https://eric-corduroy.github.io/geospatial-narratives/index.html) vor, die
  die oft fehlenden Metadaten von Satellitenbildern in den Fokus rückt und sich als offenes,
  kuratorisches sowie methodisches Werkzeug zur Sammlung, Kontextualisierung und medienkritischen
  Einordnung versteht.%
}%


%%%%%%%%%%%%%%%%%%%%%%%%%%%%%%%%%%%%%%%%%%%

% time: Thursday 19:00
% URL: https://pretalx.com/fossgis2023/talk/fossgis2026-85176-mitgliederversammlung-fossgis-e-v/

%
\newTimeslot{19:00}
\noindent\abstractHSzwei{%
  %
}{%
  Mitgliederversammlung FOSSGIS e.V.%
}{%
}{%
  Die jährliche Mitgliederversammlung findet im Rahmen der FOSSGIS Konferenz in Göttingen statt. Die
  Einladung ging an alle Mitglieder im Voraus per Mail. Auch auf der FOSSGIS neu beigetretenen
  Miglieder können natürlich teilnehmen.%
}%


%%%%%%%%%%%%%%%%%%%%%%%%%%%%%%%%%%%%%%%%%%%
