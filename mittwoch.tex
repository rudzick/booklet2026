
% time: Wednesday 10:00
% URL: https://pretalx.com/fossgis2023/talk/fossgis2026-85173-eroffnung/

%
\newTimeslot{10:00}
\noindent\abstractHSeins{%
  %
}{%
  Eröffnung%
}{%
}{%
  Feierliche Eröffnung der Konferenz durch Vertreter des FOSSGIS e.V. mit wertvollen Hinweisen zum
  Ablauf und der Organisation.%
}%


%%%%%%%%%%%%%%%%%%%%%%%%%%%%%%%%%%%%%%%%%%%

% time: Wednesday 10:30
% URL: https://pretalx.com/fossgis2023/talk/fossgis2026-84132-digitale-souveranitat-und-open-source/

%
\newTimeslot{10:30}
\noindent\abstractHSeins{%
  Olaf Knopp%
}{%
  Digitale Souveränität und Open Source%
}{%
}{%
  Angesichts der aktuellen Herausforderungen wird dargestellt, welche Bedeutung Digitale
  Souveränität für Behörden und die freie Wirtschaft hat und warum Freie und Open Source Software
  ein Garant für Datensicherheit in unsicheren Zeiten ist.%
}%


%%%%%%%%%%%%%%%%%%%%%%%%%%%%%%%%%%%%%%%%%%%

% time: Wednesday 10:35
% URL: https://pretalx.com/fossgis2023/talk/fossgis2026-83970-daten-und-prozesssicherheit-in-der-cloud-wo-sind-meine-daten-und-wer-hat-zugriff/

%
\newTimeslot{10:35}
\noindent\abstractHSeins{%
  Torsten Wiebke, Torsten Friebe, David Arndt%
}{%
  Daten- und Prozesssicherheit in der Cloud~-- Wo sind meine Daten und wer hat Zugriff?%
}{%
}{%
  Am Beispiel eines Szenarios~-- vom Weg einer politischen Entscheidung bis zur Änderung des
  Flächennutzungs- und Bebauungsplanes sowie der Kartierung und fachlichen Stellungnahme –
  diskutieren wir in einem Panelgespräch die Notwendigkeit digitaler Souveränität.
  Im Mittelpunkt stehen Handlungsfähigkeit, Vertrauen und zukunftssichere Kontrolle über Geodaten
  und Verfahren~-- insbesondere in der öffentlichen Verwaltung.%
}%


%%%%%%%%%%%%%%%%%%%%%%%%%%%%%%%%%%%%%%%%%%%

% time: Wednesday 11:45
% URL: https://pretalx.com/fossgis2023/talk/fossgis2026-83197-von-der-datensatz-zur-modulbasierten-datenintegration-mit-der-udp-data-automation/

%
\newTimeslot{11:45}
\noindent\abstractHSdrei{%
  Sarah Spönemann, Katharina Lupp%
}{%
  Von der datensatz- zur modulbasierten Datenintegration mit der UDP Data Automation%
}{%
}{%
  Die Urban Data Platform Hamburg veröffentlicht über 600 Datensätze aus unterschiedlichen Bereichen
  der Stadt. Um die Entwicklung und Pflege der datensatzbezogenen, auf proprietärer Software
  basierenden Datenintegrationsprozesse zu vereinfachen, wird ein modularer und dynamischer
  ETL-Prozess auf Basis von Open Source Technologien, wie Apache Airflow und Python, entwickelt. Mit
  der UDP Data Automation werden Datenintegrationsprozesse künftig effizienter, flexibler und
  nachhaltiger.%
}%


%%%%%%%%%%%%%%%%%%%%%%%%%%%%%%%%%%%%%%%%%%%

% time: Wednesday 11:45
% URL: https://pretalx.com/fossgis2023/talk/fossgis2026-82470-xmas-plugin-xplanung-und-andere-anwendungsschemas-mit-qgis/

%

\noindent\abstractHSvier{%
  Tobias Kraft%
}{%
  XMAS-Plugin~-- XPlanung und andere Anwendungsschemas mit QGIS%
}{%
}{%
  Aufbauend auf der Python-Bibliothek
  [XPlan-Tools](https://gitlab.opencode.de/xleitstelle/xplanung/xplan-tools) wurde mit dem
  XLeitstelle Modellgetriebenen AnwendungsSchema (XMAS) Plugin eine QGIS-basierte Open Source Lösung
  zur Visualisierung, Digitalisierung und Bearbeitung von Plänen nach Standards der XLeitstelle -
  insbesondere XPlanung~-- entwickelt.  Dabei wird das vollständige Datenmodell sowie der Im- und
  Export von GML-Dateien unterstützt.%
}%


%%%%%%%%%%%%%%%%%%%%%%%%%%%%%%%%%%%%%%%%%%%

% time: Wednesday 12:20
% URL: https://pretalx.com/fossgis2023/talk/fossgis2026-83971-sichere-softwarelieferketten-mit-opencode/

%
\newTimeslot{12:20}
\noindent\abstractHSeins{%
  David Arndt, Florian Micklich, Torsten Friebe, Alexandros Bouras (ZenDIS)%
}{%
  Sichere Softwarelieferketten mit openCode%
}{%
}{%
  openCode ist die gemeinsame Plattform der Öffentlichen Verwaltung für die Erstellung,
  Qualitätssicherung, Austausch und Nachnutzung von OSS. Mit dem Badge Programm und den Plattform
  Services wurde in 2025 das Angebot für Bund, Land und Kommunen ausgebaut, um vertrauenswürdige OSS
  glaubhaft zu attestieren. Mit der souveränen Softwarelieferkette demonstriert openCode, wie OSS
  das IT-Sicherheitsniveau durch sichere Softwareprüfung, dezentrale Container-Registry etc.
  nachhaltig etabliert.%
}%


%%%%%%%%%%%%%%%%%%%%%%%%%%%%%%%%%%%%%%%%%%%

% time: Wednesday 12:20
% URL: https://pretalx.com/fossgis2023/talk/fossgis2026-84273-harmonisierung-heterogener-geodaten-mit-open-source-pipelines/

%

\noindent\abstractHSzwei{%
  Jens Wiesehahn, David Kunze, Florian Franz%
}{%
  Harmonisierung heterogener Geodaten mit Open-Source-Pipelines%
}{%
}{%
  Luftbilder und LiDAR-Daten variieren stark in Format, Qualität und Zugänglichkeit, was ihre
  interoperable Nutzung über große Gebiete erheblich erschwert. Um diese Heterogenität zu
  reduzieren, wurden automatisierte Verarbeitungspipelines auf Basis freier Software entwickelt. So
  entstehen standardisierte, bereinigte und Cloud-optimierte Datensätze für effiziente Analysen.%
}%


%%%%%%%%%%%%%%%%%%%%%%%%%%%%%%%%%%%%%%%%%%%

% time: Wednesday 12:20
% URL: https://pretalx.com/fossgis2023/talk/fossgis2026-83595-von-der-quelle-zur-api-standardisierte-nahe-echtzeit-datenintegration-in-die-udp/

%

\noindent\abstractHSdrei{%
  Finn Jorczik%
}{%
  Von der Quelle zur API: Standardisierte Nahe-Echtzeit-Datenintegration in die UDP%
}{%
}{%
  Die Bereitstellung von Sensordaten in Nahe-Echtzeit stellt kommunale Verwaltungen wie die Stadt
  Hamburg vor neue Herausforderungen. Der hier vorgestellte Prototyp eines Standard-Konnektors
  vereinfacht die Integration und die Veröffentlichung solcher Daten im OGC-Standard Sensor Things
  API. Im Vortrag werden der Einsatz des Prototyps im laufenden Betrieb sowie aktuelle
  Weiterentwicklungen zum Erreichen der vollständigen Betriebsfähigkeit und Nachnutzbarkeit des
  Konnektors vorgestellt.%
}%


%%%%%%%%%%%%%%%%%%%%%%%%%%%%%%%%%%%%%%%%%%%

% time: Wednesday 14:15
% URL: https://pretalx.com/fossgis2023/talk/fossgis2026-84052-open-data-und-open-source-strategie-des-bkg/

%
\newTimeslot{14:15}
\noindent\abstractHSeins{%
  Konstantin Krömer, Florian Micklich, Mariela Masino%
}{%
  Open Data und Open Source-Strategie des BKG%
}{%
}{%
  Das BKG hat ein breites Open Data-Angebot, welches sich stetig erweitert. Was hat sich seitdem
  getan? Welche Datensätze sind neu hinzugekommen? Und wie steht es um das Thema Open Source?
  Im Vortrag werden die Neuerungen im Bereich Open Data vorgestellt. Dabei soll neben den
  Datensätzen auch auf Neuerungen in Bezug auf die verwendeten Lizenzen eingegangen werden.
  Ergänzend wird der Status Quo von Open Source am Bundesamt vorgestellt und ein Ausblick in die
  Zukunft gegeben.%
}%


%%%%%%%%%%%%%%%%%%%%%%%%%%%%%%%%%%%%%%%%%%%

% time: Wednesday 14:15
% URL: https://pretalx.com/fossgis2023/talk/fossgis2026-83749-ein-offenes-dezentrales-netzwerk-fur-metadaten/

%

\noindent\abstractHSzwei{%
  Volker Mische%
}{%
  Ein offenes dezentrales Netzwerk für Metadaten%
}{%
}{%
  Ein offener, dezentraler Ansatz zur Speicherung von Metainformationen zu Geodaten ermöglicht es
  kleineren Akteuren, wie NGOs, Forschungsteams und Privatpersonen, ihre erhobenen oder abgeleiteten
  Daten leicht auffindbar und nachhaltig bereitzustellen. Der Ansatz beruht auf modernen, offenen
  (cloud-optimized) Datenformaten und Protokollen wie GeoZarr, Icechunk, Apache Iceberg, DASL und
  ATProto, die Interoperabilität, Integrität und Skalierbarkeit sicherstellen.%
}%


%%%%%%%%%%%%%%%%%%%%%%%%%%%%%%%%%%%%%%%%%%%

% time: Wednesday 14:15
% URL: https://pretalx.com/fossgis2023/talk/fossgis2026-84222-qgis-web-client-anwendertreffen/

%

\noindent\abstractOther{%
  Sandro Mani%
}{%
  QGIS Web Client Anwendertreffen%
}{%
}{%
  Das Treffen soll QWC-Anwendern und -Administratoren die Möglichkeit geben, eigene Erfahrungen mit
  anderen Anwendern zu teilen und neue Kontakte zu knüpfen. Teilnehmer können ihre eigenen, mit QWC
  realisierten WebGIS-Projekte vorstellen und gemeinsam evtl. auftretende Probleme diskutieren oder
  anderen Tipps geben.
  Alle Interessenten sind herzlich eingeladen, beim Anwendertreffen vorbeizuschauen!%
}%
{%
  BoF1 (ZHG 001)%
}%



%%%%%%%%%%%%%%%%%%%%%%%%%%%%%%%%%%%%%%%%%%%

% time: Wednesday 14:15
% URL: https://pretalx.com/fossgis2023/talk/fossgis2026-83714-postnas-suite-anwendertreffen/

%

\noindent\abstractOther{%
  Astrid Emde%
}{%
  PostNAS-Suite Anwendertreffen%
}{%
}{%
  Die PostNAS-Suite Anwender:innen kommunizieren über die Mailingliste und treffen sich zum
  Austausch. Das nächste Treffen soll auf der FOSSGIS 2026 stattfinden. Hier sollen aktuelle
  Entwicklungen im PostNAS-Suite Projekt vorgestellt und die Erfahrungen der Anwender:innen
  ausgetauscht werden.%
}%
{%
  BoF2 (ZHG 005)%
}%



%%%%%%%%%%%%%%%%%%%%%%%%%%%%%%%%%%%%%%%%%%%

% time: Wednesday 14:15
% URL: https://pretalx.com/fossgis2023/talk/fossgis2026-83968-opencode-de-softwarequalitat-erkennen-badges/

%

\noindent\abstractAnwBoFdrei{%
  David Arndt, Torsten Friebe, Alexandros Bouras (ZenDIS)%
}{%
  OpenCode.de: Softwarequalität erkennen~-- Badges%
}{%
}{%
  openCode prüft öffentliche Softwareprojekte nach definierten Sicherheits-, Wartungs- und
  Nachnutzungskriterien und vergibt darauf basierend automatisierte Qualitätssiegel. Diese
  ermöglichen es Behörden, auf einen Blick nachvollziehbare Qualitätsaussagen zu treffen. Das
  Badge-Programm stellt dabei einen Paradigmenwechsel dar: Statt aufwendiger Einzelfallprüfungen
  entstehen skalierbare, transparente Bewertungsverfahren. Dies schafft Vertrauen in OSS-Einsatz
  innerhalb der öffentlichen Verwaltung.%
}%


%%%%%%%%%%%%%%%%%%%%%%%%%%%%%%%%%%%%%%%%%%%

% time: Wednesday 14:50
% URL: https://pretalx.com/fossgis2023/talk/fossgis2026-84237-open-data-im-umweltbundesamt/

%
\newTimeslot{14:50}
\noindent\abstractHSeins{%
  Luise Quoß%
}{%
  Open Data im Umweltbundesamt%
}{%
}{%
  (Unvollständige) Open Data und deren (erschwerte) Auffindbarkeit in öffentlichen Behörden ist ein
  regelmäßiges Thema auf der FOSSGIS. Das Umweltbundesamt arbeitet auf der Grundlage seiner
  Datenstrategie an der Umsetzung einer datenzentrierten Denk- und Arbeitsweise. Ein zentraler
  Baustein in diesem Prozess ist der Aufbau eines Metadatenkatalog für die vielfältigen Umwelt- und
  Naturschutz-Daten. Wir stellen den aktuellen Status Quo des Metadatenkatalogs und seine Rolle in
  der Datenstrategie vor.%
}%


%%%%%%%%%%%%%%%%%%%%%%%%%%%%%%%%%%%%%%%%%%%

% time: Wednesday 14:50
% URL: https://pretalx.com/fossgis2023/talk/fossgis2026-84115-datenkataloge-mit-stac-und-ogc-api-records/

%

\noindent\abstractHSzwei{%
  Pirmin Kalberer%
}{%
  Datenkataloge mit STAC und OGC API Records%
}{%
}{%
  STAC (SpatioTemporal Asset Catalogs) ist ein Standard für die Organisation und Beschreibung
  geografischer und zeitlicher Daten. Er wird von zahlreichen Organisation und Projekten verwendet
  und hat eine breite Toolunterstützung.
  Der OGC Standard für Metadatendienste CSW wurde kürzlich durch OGC API~-- Records abgelöst, welcher
  grosse Ähnlichkeiten mit STAC aufweist.%
}%


%%%%%%%%%%%%%%%%%%%%%%%%%%%%%%%%%%%%%%%%%%%

% time: Wednesday 14:50
% URL: https://pretalx.com/fossgis2023/talk/fossgis2026-84303-einsatz-von-fossgis-und-offenen-geodaten-in-der-forstlichen-lehre/

%

\noindent\abstractHSvier{%
  Paul Magdon%
}{%
  Einsatz von FOSSGIS und offenen Geodaten in der forstlichen Lehre%
}{%
}{%
  Der Beitrag zeigt am Beispiel forstlicher Studiengänge an der HAWK Göttingen, wie FOSSGIS, Open
  Data und moderne Hardware (z.B. GNSS, Drohnen) erfolgreich in Vorlesungen, Übungen und Prüfungen
  integriert werden. Durch praxisnahe Projekte werden Fachkräfte ausgebildet, die digitale
  Kompetenzen und den souveränen Umgang mit offenen Geodaten und Tools für Wissenschaft und Praxis
  erlernen.%
}%


%%%%%%%%%%%%%%%%%%%%%%%%%%%%%%%%%%%%%%%%%%%

% time: Wednesday 15:25
% URL: https://pretalx.com/fossgis2023/talk/fossgis2026-84233-mit-open-source-zum-lakehouse-cluster-mit-nativer-geo-unterstutzung/

%
\newTimeslot{15:25}
\noindent\abstractHSdrei{%
  Gabriel Musial%
}{%
  Mit Open Source zum Lakehouse-Cluster mit nativer Geo-Unterstützung%
}{%
}{%
  Der Vortrag vermittelt Motivation und Orientierung für den Aufbau eines eigenen
  Lakehouse-Clusters. Er erklärt kompakt Architektur und Kernbausteine eines Lakehouse-Systems~-- vom
  Katalog über Speicher und Compute bis zum Tabellenformat~-- und zeigt, wie man einen Plan für große
  Datenmengen entwickelt, wenn eine einzelne Maschine nicht mehr ausreicht.%
}%


%%%%%%%%%%%%%%%%%%%%%%%%%%%%%%%%%%%%%%%%%%%

% time: Wednesday 15:25
% URL: https://pretalx.com/fossgis2023/talk/fossgis2026-82776-mercator-ist-dein-freund/

%

\noindent\abstractHSvier{%
  Javier Jimenez Shaw%
}{%
  Mercator ist dein Freund%
}{%
}{%
  Die Mercator-Projektion wird heutzutage von vielen verachtet. Dies geht auf die Kritik  von Arno
  Peters aus den (mindestens) 1970er Jahren zurück. Die Mercator-Projektion verfügt jedoch über
  einzigartige Eigenschaften.
  Dieser Vortrag erläutert die Vor- und Nachteile der Mercator-Projektion und vergleicht sie mit
  anderen Projektionen.
  Die Mercator-Projektion ist nicht schlecht. Mercator ist dein Freund. Man muss nur wissen, wann
  man sie einsetzt und wann nicht.%
}%


%%%%%%%%%%%%%%%%%%%%%%%%%%%%%%%%%%%%%%%%%%%

% time: Wednesday 16:30
% URL: https://pretalx.com/fossgis2023/talk/fossgis2026-84034-wir-spielen-mit-offenen-karten-der-fossgis-e-v/

%
\newTimeslot{16:30}
\noindent\abstractHSeins{%
  Katja Haferkorn, Jochen Topf%
}{%
  Wir spielen mit offenen Karten~-- Der FOSSGIS e.V.%
}{%
}{%
  Der Vortrag gibt einen Blick "`hinter die Kulissen"' für alle, die schon immer mal wissen wollten,
  wie die Vereinsarbeit gemacht wird. Und für diejenigen, die mitmachen wollen, aber noch nicht so
  richtig wissen wie und wo.%
}%


%%%%%%%%%%%%%%%%%%%%%%%%%%%%%%%%%%%%%%%%%%%

% time: Wednesday 16:30
% URL: https://pretalx.com/fossgis2023/talk/fossgis2026-84143-volltextsuche-in-echtzeitdaten-mit-pgsearch/

%

\noindent\abstractHSdrei{%
  Marco Hugentobler%
}{%
  Volltextsuche in Echtzeitdaten mit pg\_search%
}{%
}{%
  pg\_search ist eine neue PostgreSQL-Erweiterung für die Volltextsuche mit dem BM25 Algorithmus.
  Damit können gleichwertige Ergebnisse wie mit externen Suchmaschinen (Elasticsearch, Solr)
  erreicht werden.
  Dank der automatischen Aktualisierung des Suchindexes ermöglicht pg\_search den sofortigen Einbezug
  von neuen oder aktualisierten Daten.%
}%


%%%%%%%%%%%%%%%%%%%%%%%%%%%%%%%%%%%%%%%%%%%

% time: Wednesday 16:30
% URL: https://pretalx.com/fossgis2023/talk/fossgis2026-84316-webgis-framework-fur-multidimensionale-potenzialanalyse-nachhaltiger-warmeversorgung/

%

\noindent\abstractHSvier{%
  Abdulraheem Salaymeh%
}{%
  WebGIS-Framework für multidimensionale Potenzialanalyse nachhaltiger Wärmeversorgung%
}{%
}{%
  Die Transformation zur nachhaltigen Wärmeversorgung erfordert gezielte Analyse und Integration
  zahlreicher Daten. Der HAWK Kompass unterstützt die Wärmeplanung in Bestands- und
  Potenzialanalysen, indem Open Data zahlreicher Quellen in ein WebGIS-Energy-Framework eingebunden
  werden. Die integrierten Funktionen unterstützen interaktive und anpassbare Quartiersanalysen und
  Datenexport. Planer:innen erkennen regionale Energie- und Infrastrukturpotenziale schnell und
  treffen fundierte Entscheidungen.%
}%


%%%%%%%%%%%%%%%%%%%%%%%%%%%%%%%%%%%%%%%%%%%

% time: Wednesday 16:30
% URL: https://pretalx.com/fossgis2023/talk/fossgis2026-84335-anwendertreffen-qgis-in-forst-landwirtschaft-und-grunem-bereich/

%

\noindent\abstractOther{%
  Daniel Keune, Janine Raatz%
}{%
  Anwendertreffen QGIS in Forst-, Landwirtschaft und "`Grünem"' Bereich%
}{%
}{%
  Das Anwendertreffen zur Nutzung von QGIS in Forst- und Landwirtschaft soll ein Treffen zum
  Erfahrungsaustausch, Vernetzen und gegenseitigen Lernen und kennen lernen sein.
  Teilnehmerinnen und Teilnehmer sind eingeladen, ihre eigenen GIS-Projekte aus der forstlichen und
  landwirtschaftlichen Praxis vorzustellen~-- etwa Anwendungen zur Bestandsaufnahme,
  Flächenbewirtschaftung, Holz- oder Erntelogistik, Boden- und Standortanalyse,
  Biodiversitätskartierung oder Planung von Pflegemaßnahmen.%
}%
{%
  BoF1 (ZHG 001)%
}%



%%%%%%%%%%%%%%%%%%%%%%%%%%%%%%%%%%%%%%%%%%%

% time: Wednesday 17:05
% URL: https://pretalx.com/fossgis2023/talk/fossgis2026-84329-warum-es-eine-gute-idee-war-eine-fossgis-firma-zu-grunden/

%
\newTimeslot{17:05}
\noindent\abstractHSeins{%
  Jens Fitzke%
}{%
  Warum es eine gute Idee war, eine FOSSGIS-Firma zu gründen%
}{%
}{%
  Der Beitrag mit dem Untertitel "`... und warum ich es heute nicht noch einmal machen würde~-- oder
  vielleicht doch?"' bietet einen selbstkritischen Rückblick auf die 25-jährige Geschäftstätigkeit
  von lat/lon, einer Firma mit einem auf dem Open Source-Gedanken basierenden Businessmodell im
  Bereich der Geo-IT, und beleuchtet die Herausforderungen, die Chancen und Risiken, früher und
  heute, und was es unterwegs zu lernen gab, mit dem Ziel, die Erfahrungen praxisnah zu
  präsentieren.%
}%


%%%%%%%%%%%%%%%%%%%%%%%%%%%%%%%%%%%%%%%%%%%

% time: Wednesday 17:05
% URL: https://pretalx.com/fossgis2023/talk/fossgis2026-84259-das-masterportal-in-einer-behordlichen-cloudumgebung/

%

\noindent\abstractHSzwei{%
  Marko Neukamm%
}{%
  Das Masterportal in einer behördlichen Cloudumgebung%
}{%
}{%
  Die GDI-Th wird modernisiert. Kernstück ist der Geoproxy 4.0 mit cloud- und containerbasierter
  Architektur. Als zentrale Darstellungskomponente dient der Thüringen Viewer auf Basis des
  Masterportal-Frameworks. Der Vortrag behandelt Herausforderungen der Umsetzung in der Thüringer
  Verwaltungscloud unter Berücksichtigung der Rahmenbedingungen des Landesrechenzentrums sowie die
  Integration in bestehende Infrastrukturen wie die landesinterne Nutzerverwaltung und die Anbindung
  an die BundID.%
}%


%%%%%%%%%%%%%%%%%%%%%%%%%%%%%%%%%%%%%%%%%%%

% time: Wednesday 17:05
% URL: https://pretalx.com/fossgis2023/talk/fossgis2026-83926-ein-ansatz-zur-optimierung-von-raumbezogenen-datenbankabfragen-mit-postgis/

%

\noindent\abstractHSdrei{%
  Dominik Frey%
}{%
  Ein Ansatz zur Optimierung von raumbezogenen Datenbankabfragen mit PostGIS%
}{%
}{%
  In diesem Vortrag wird gezeigt, wie mit der PostgreSQL-Erweiterung PostGIS räumliche Daten
  effizient verarbeitet werden können. Anhand eines konkreten Anwendungsbeispiels wird erläutert,
  wie ein optimiertes PostGIS-Datenbankschema und maßgeschneiderte Funktionen eine schnelle
  Visualisierung von Echtzeitinformationen in Katastrophengebieten ermöglichen und so eine effektive
  Entscheidungsfindung unterstützen.%
}%


%%%%%%%%%%%%%%%%%%%%%%%%%%%%%%%%%%%%%%%%%%%

% time: Wednesday 17:05
% URL: https://pretalx.com/fossgis2023/talk/fossgis2026-82771-nutzung-raumlicher-daten-fur-eine-vertragliche-vertikale-nachverdichtung/

%

\noindent\abstractHSvier{%
  Anne Fischer%
}{%
  Nutzung räumlicher Daten für eine verträgliche vertikale Nachverdichtung%
}{%
}{%
  Jährlich sollen 400.000 neue Wohnungen geschaffen werden unter Beibehaltung des Flächensparziels.
  Eine Lösungsstrategie ist die vertikale Nachverdichtung von Gebäuden. Doch ein Eingriff in den
  Siedlungsbestand ist oft konfliktbehaftet. Daher werden anerkannte städtebauliche Kriterien in
  räumliche Daten übersetzt und verwendet, um verträgliche Nachverdichtungspotenziale an einem
  Fallbeispiel zu ermitteln. Hierdurch soll ein objektiver Ansatz für eine effiziente
  Innenentwicklung geschaffen werden.%
}%


%%%%%%%%%%%%%%%%%%%%%%%%%%%%%%%%%%%%%%%%%%%

% time: Wednesday 17:40
% URL: https://pretalx.com/fossgis2023/talk/fossgis2026-84130-cloud-native-geodateninfrastrukturen-skalierbarer-open-source-ansatz-mit-kubernetes/

%
\newTimeslot{17:40}
\noindent\abstractHSzwei{%
  Charlotte Toma, Alexander Rolwes, Lisa Dannhorn, Katja Albrecht%
}{%
  Cloud-native Geodateninfrastrukturen: Skalierbarer Open-Source-Ansatz mit Kubernetes%
}{%
}{%
  Der Vortrag beschreibt die Migration der GDI des Stadtplanungsamts der LH Wiesbaden von einer
  klassischen, serverbasierten Umgebung hin zu einer cloud-nativen Architektur mit Kubernetes. Als
  leistungsstarke Container-Orchestrierungsplattform ermöglicht es die automatisierte Verwaltung,
  Skalierung und Überwachung unserer auf Open-Source ausgerichteter Geodaten-Services. Wir zeigen
  exemplarisch, wie Kommunen mit OSS und moderner Infrastruktur ihre Geodaten effizient und flexibel
  verwalten können.%
}%


%%%%%%%%%%%%%%%%%%%%%%%%%%%%%%%%%%%%%%%%%%%

% time: Wednesday 18:15
% URL: https://pretalx.com/fossgis2023/talk/fossgis2026-84196-geodaten-management-mit-postgis/

%
\newTimeslot{18:15}
\noindent\abstractHSdrei{%
  Marion Baumgartner%
}{%
  Geodaten-Management mit PostGIS%
}{%
}{%
  Wie lassen sich Geodaten mit PostGIS effizient speichern und nutzen? In diesem Vortrag zeige ich
  verschiedene Möglichkeiten, räumliche Daten performant zu speichern und zu visualisieren. Anhand
  laufender Projekte und anhand von Beispielen werden gängige Datenmodelle vorgestellt und ihre
  jeweiligen Vor- und Nachteile präsentiert.%
}%


%%%%%%%%%%%%%%%%%%%%%%%%%%%%%%%%%%%%%%%%%%%

% time: Wednesday 18:15
% URL: https://pretalx.com/fossgis2023/talk/fossgis2026-84348-parkingspacepotential-platz-machen-fur-die-verkehrswende/

%

\noindent\abstractHSvier{%
  Lisa-Marie Jalyschko%
}{%
  ParkingSpacePotential~-- Platz machen für die Verkehrswende%
}{%
}{%
  Sichere Infrastruktur für Fuß- und Radverkehr, mehr Straßengrün oder Sitzmöglichkeiten: All das
  braucht Platz im öffentlichen Straßenraum. Dieser wird häufig ganz selbstverständlich von Kfz
  zugeparkt. Gleichzeitig lässt sich beobachten, wie Parkhäuser leer stehen. Das QGIS-Plugin
  ParkingSpacePotential zeigt auf, welche Parkplätze aus dem öffentlichen Straßenraum in ein
  Parkhaus verlagert werden könnten.%
}%


%%%%%%%%%%%%%%%%%%%%%%%%%%%%%%%%%%%%%%%%%%%
