
% time: Friday 09:00
% URL: https://pretalx.com/fossgis2023/talk/fossgis2026-83258-schnellere-karten-kleinere-maplibre-vektorkacheln-und-weitere-neuerungen/

%
\newTimeslot{09:00}
\noindent\abstractHSeins{%
  Frank Elsinga, Bart Louwers%
}{%
  Schnellere Karten, kleinere MapLibre Vektorkacheln und weitere Neuerungen%
}{%
}{%
  MapLibre steht für offene Innovation im Bereich vektorbasiertes Kartenrendering. In diesem Vortrag
  zeigt, wie das Open-Source-Ökosystem rund um MapLibre immer neue Impulse setzt.
  Der Talk bietet einen breiten Überblick über aktuelle Entwicklungen, erklärt die wichtigsten
  Konzepte für Einsteige\textbackslash1(\textbackslash2) und gibt einen Ausblick, wohin sich offene Kartentechnologien in den
  nächsten Jahren bewegen.%
}%


%%%%%%%%%%%%%%%%%%%%%%%%%%%%%%%%%%%%%%%%%%%

% time: Friday 09:00
% URL: https://pretalx.com/fossgis2023/talk/fossgis2026-82939-routing-mit-motis/

%

\noindent\abstractHSdrei{%
  Dr. Felix Gündling, Robin Durner%
}{%
  Routing mit MOTIS%
}{%
}{%
  MOTIS ist eine quelloffene intermodale Mobilitätsplattform, die alle Verkehrsarten (öffentlicher
  Verkehr, On-Demand Verkehr, Micro/Sharing Mobility, private Mitnahme, usw.) miteinander verknüpft.
  In diesem Vortrag werden aktuelle Entwicklungen in MOTIS vorgestellt, z.B. Funktionen im Bereich
  Preissysteme (GTFS Fares v2), die Unterstützung von On-Demand Systemen (GTFS Flex),
  Ereignismeldungen (GTFS Service Alerts), Verbesserungen im geocoding, und neue Scripting
  Funktionen.%
}%


%%%%%%%%%%%%%%%%%%%%%%%%%%%%%%%%%%%%%%%%%%%

% time: Friday 09:00
% URL: https://pretalx.com/fossgis2023/talk/fossgis2026-82723-mentoring-programm-des-fossgis-vereins-gemeinsam-ankommen-gemeinsam-wachsen/

%

\noindent\abstractOther{%
  Janine Raatz%
}{%
  Mentoring-Programm des FOSSGIS-Vereins~-- Gemeinsam ankommen, gemeinsam wachsen%
}{%
}{%
  Wie kann der Verein neue Mitglieder auf ihrem Weg begleiten? In dieser Session stellen sich unsere
  Mentoring Buddys vor und zeigen, wie das Buddy-Programm hilft, gut anzukommen, Fragen zu klären
  und echte Verbindungen zu schaffen.
  Für alle, die neu dabei sind~-- und alle, die Lust haben, selbst Buddy zu werden!%
}%
{%
  BoF1 (ZHG 001)%
}%



%%%%%%%%%%%%%%%%%%%%%%%%%%%%%%%%%%%%%%%%%%%

% time: Friday 09:00
% URL: https://pretalx.com/fossgis2023/talk/fossgis2026-84056-anwendertreffen-lizmap-webclient/

%

\noindent\abstractOther{%
  Günter Wagner%
}{%
  Anwendertreffen Lizmap-Webclient%
}{%
}{%
  Die deutschsprachige Anwendergruppe für den WebClient Lizmap nutzt das jährliche Treffen auf der
  FOSSGIS zum Erfahrungsaustausch und zur Kontaktpflege.
  Teilnehmer können ihre eigenen, mit QGIS-Server und Lizmap realisierten WebGIS-Projekte
  vorstellen. Ferner kann über aktuelle Fragen/Probleme und zukünftige, gewünschte Erweiterungen in
  Lizmap diskutiert werden.
  Das Anwendertreffen richtet sich sowohl an neu Interessierte, als auch an Anwender, die bereits
  mit Lizmap arbeiten.%
}%
{%
  BoF2 (ZHG 005)%
}%



%%%%%%%%%%%%%%%%%%%%%%%%%%%%%%%%%%%%%%%%%%%

% time: Friday 09:35
% URL: https://pretalx.com/fossgis2023/talk/fossgis2026-83719-mapbender-in-neuem-design-und-unterstutzung-weiterer-datenquellen/

%
\newTimeslot{09:35}
\noindent\abstractHSeins{%
  Astrid Emde, Thorsten Hack%
}{%
  Mapbender in neuem Design und Unterstützung weiterer Datenquellen%
}{%
}{%
  Auch im letzten Jahr ist im Mapbender-Projekt wieder viel passiert. Es soll von den Neuigkeiten im
  Projekt berichtet werden und neue Features sollen vorgestellt.
  Der Schwerpunkt lag dabei auf der Überarbeitung des Designs und der Unterstützung weiterer
  Datenquellen. Aber auch viele kleine Neuerungen sind hinzugekommen, die vorgestellt werden sollen.
  Die Neuerungen werden anhand von einigen Mapbender-Projekten vorgestellt.%
}%


%%%%%%%%%%%%%%%%%%%%%%%%%%%%%%%%%%%%%%%%%%%

% time: Friday 09:35
% URL: https://pretalx.com/fossgis2023/talk/fossgis2026-84267-arbeit-mit-standardisierten-karten-dynamische-kartenerstellung-mit-pyqgis/

%

\noindent\abstractHSvier{%
  Laura Vecera, Jonas Danner%
}{%
  Arbeit mit standardisierten Karten~-- Dynamische Kartenerstellung mit PyQGIS%
}{%
}{%
  Wer häufig Karten in QGIS erstellt, ist mit Drucklayouts und Atlanten vertraut. In dieser
  Demo-Session stellen wie vor, wie mit einer Python-Automatisierung die Kartenerstellung effizient
  und professionell standardisiert werden kann.  Durch den Einsatz von Variablen, Funktionen und
  Expressions in PyQGIS werden Kartenelemente wie Legende, Titel und Metadaten dynamisch an ihren
  Inhalt angepasst und in Relation zu anderen Kartenelementen neu positioniert.%
}%


%%%%%%%%%%%%%%%%%%%%%%%%%%%%%%%%%%%%%%%%%%%

% time: Friday 10:10
% URL: https://pretalx.com/fossgis2023/talk/fossgis2026-84218-qgis-web-client-neues-aus-dem-projekt/

%
\newTimeslot{10:10}
\noindent\abstractHSeins{%
  Sandro Mani%
}{%
  QGIS Web Client~-- Neues aus dem Projekt%
}{%
}{%
  Der QGIS Web Client (QWC) ist eine ausgereifte Anwendung zur Publikation von QGIS Projekte in Web.
  Sie bietet sowohl eine 2D wie auch eine 3D Ansicht.
  Dieser Vortrag gibt einen Überblick über die Architektur des QWC, und stellt die zahlreichen neuen
  Funktionen vor, die im letzten Jahr entwickelt wurden.%
}%


%%%%%%%%%%%%%%%%%%%%%%%%%%%%%%%%%%%%%%%%%%%

% time: Friday 10:10
% URL: https://pretalx.com/fossgis2023/talk/fossgis2026-84252-erfahrungsbericht-mergin-maps-eine-neue-open-source-losung-fur-felddatenerfassung/

%

\noindent\abstractHSzwei{%
  Geneviève Hannes, Julian Hafner%
}{%
  Erfahrungsbericht Mergin Maps: eine 'neue' Open-Source-Lösung für Felddatenerfassung%
}{%
}{%
  Mergin Maps ist eine in den letzten Jahren in Europa entwickelte Open-Source-Anwendung, welche die
  effiziente Erfassung räumlicher Daten im Feld mithilfe von mobilen Geräten (Smartphones / Tablets)
  ermöglicht.
  Da Camptocamp Mergin Maps bereits erfolgreich in verschiedenen Projekten in Frankreich und in der
  Schweiz eingesetzt hat, möchten wir euch einige diese Praxisbeispiele näher vorstellen.%
}%


%%%%%%%%%%%%%%%%%%%%%%%%%%%%%%%%%%%%%%%%%%%

% time: Friday 10:10
% URL: https://pretalx.com/fossgis2023/talk/fossgis2026-84139-openstreetmap-daten-als-grundlage-fur-routing-wie-gut-funktioniert-das-wirklich/

%

\noindent\abstractHSdrei{%
  Benjamin Würzler, Benjamin Herfort%
}{%
  OpenStreetMap-Daten als Grundlage für Routing~-- wie gut funktioniert das wirklich?%
}{%
}{%
  Das HeiGIT und das BKG untersuchen, wie die Qualität von OSM-Daten Routingentscheidungen
  beeinflusst. Mit der ohsome quality API werden Vollständigkeit, Attributgenauigkeit und Aktualität
  der Daten analysiert und mit Routingvergleichen zwischen Openrouteservice, Google Maps, Bing Maps
  und Apple Maps verknüpft. Die Ergebnisse zeigen robuste OSM-Leistungen und liefern praxisrelevante
  Erkenntnisse zur Nutzung und Bewertung von OSM-Daten für Routing.%
}%


%%%%%%%%%%%%%%%%%%%%%%%%%%%%%%%%%%%%%%%%%%%

% time: Friday 11:10
% URL: https://pretalx.com/fossgis2023/talk/fossgis2026-84262-kombination-des-openrouteservice-mit-emissionsmodelldaten-des-umweltbundesamtes/

%
\newTimeslot{11:10}
\noindent\abstractHSvier{%
  Benjamin Würzler%
}{%
  Kombination des Openrouteservice mit Emissionsmodelldaten des Umweltbundesamtes%
}{%
}{%
  Wie lassen sich Emissionen im Straßenverkehr sichtbar machen und welche Streckenabschnitte
  verursachen besonders viele? In unserer Studie zeigen wir, wie sich mit Hilfe von Daten des
  Umweltbundesamtes Emissionen entlang von Routen berechnen und visualisieren lassen~-- von der
  Herausforderung der korrekten OSM-Zuordnung, über die Verknüpfung mit dem Openrouteservice bis zu
  den ersten Ansätzen für ein nachhaltigeres Routing.%
}%


%%%%%%%%%%%%%%%%%%%%%%%%%%%%%%%%%%%%%%%%%%%

% time: Friday 11:15
% URL: https://pretalx.com/fossgis2023/talk/fossgis2026-84068-qgis-basierte-windparkplanung-fur-alle-mit-dem-saturn-planer/

%
\newTimeslot{11:15}
\noindent\abstractHSvier{%
  Christoph Neubauer%
}{%
  QGIS basierte Windparkplanung für alle mit dem SATURN Planer%
}{%
}{%
  Semi-Autmatisierte (Wind-)TURbiNenplanung. Eine kleine Reise ausgehend von den praktischen
  Problemen der Windparkplanung über einen inspirierenden FOSSGIS Vortrag bis zu einem halbwegs
  fertigen QGIS-Tool.%
}%


%%%%%%%%%%%%%%%%%%%%%%%%%%%%%%%%%%%%%%%%%%%

% time: Friday 11:45
% URL: https://pretalx.com/fossgis2023/talk/fossgis2026-84277-flurstuckssuche-in-qgis-zum-stand-eines-deutschlandweiten-ansatzes/

%
\newTimeslot{11:45}
\noindent\abstractHSzwei{%
  David Koster%
}{%
  Flurstückssuche in QGIS~-- Zum Stand eines deutschlandweiten Ansatzes%
}{%
}{%
  Orte und Adressen findet man dank OSM Place Search und ähnlichem in QGIS problemlos. Bei
  Flurstücken ist das nicht so trivial. Hier gibt es bisher höchstens Lösungen von einzelnen
  Bundesländern. Daher soll im Vortrag der aktuelle Entwicklungsstand eines QGIS-Plugins zur
  deutschlandweiten Suche von Flurstücken vorgestellt werden.%
}%


%%%%%%%%%%%%%%%%%%%%%%%%%%%%%%%%%%%%%%%%%%%

% time: Friday 11:45
% URL: https://pretalx.com/fossgis2023/talk/fossgis2026-84142-bezahltes-mapping-beispiel-barrierefreie-reisekette/

%

\noindent\abstractHSdrei{%
  Michael Reichert, Dietmar Seifert%
}{%
  Bezahltes Mapping~-- Beispiel Barrierefreie Reisekette%
}{%
}{%
  Die Nahverkehrsgesellschaft Baden-Württemberg hat im Jahr 2025 im Rahmen des Projekts
  Barrierefreie Reisekette die Erfassung von Barrierefreiheitsinformationen an ca. 1100 Bahnhöfen in
  OpenStreetMap vergeben. In diesem Vortrag berichten Auftraggeber und Auftragnehmer aus ihren
  Erfahrungen zur Vergabe, Organisation, Personalauswahl und Qualitätskontrolle bei organisierten
  Mappingkampagnen mit bezahlten Kräften.%
}%


%%%%%%%%%%%%%%%%%%%%%%%%%%%%%%%%%%%%%%%%%%%

% time: Friday 11:45
% URL: https://pretalx.com/fossgis2023/talk/fossgis2026-84146-qgis-vs-r-der-live-vergleich/

%

\noindent\abstractHSvier{%
  Oliver Archner, Stefan Holzheu%
}{%
  QGIS vs. R~-- der Live-Vergleich%
}{%
}{%
  Schrittweise Umsetzung einer räumlichen 2D-Standortanalyse gleichzeitig in R und in QGIS. Durch
  den direkten Vergleich werden sowohl Gemeinsamkeiten als auch Unterschiede erkennbar. Während die
  Analyse in R geskriptet wird, erstellen wir in QGIS ein ausführbares Modell mit Hilfe des Model
  Designers.
  Am Ende der Demo fassen wir die jeweiligen Stärken und Schwächen zusammen und geben eine
  Empfehlung, wann und wofür welches Tool die bessere Wahl ist.%
}%


%%%%%%%%%%%%%%%%%%%%%%%%%%%%%%%%%%%%%%%%%%%

% time: Friday 11:50
% URL: https://pretalx.com/fossgis2023/talk/fossgis2026-84012-sar-basiertes-monitoring-von-hangrutschungen-in-einem-tropischen-bergwald-in-ecuador/

%
\newTimeslot{11:50}
\noindent\abstractAnwBoFdrei{%
  Laura Dieter%
}{%
  SAR-basiertes Monitoring von Hangrutschungen in einem tropischen Bergwald in Ecuador%
}{%
}{%
  Die Projektarbeit untersucht die Detektion von Hangrutschungen in des tropischen Bergwaldes
  in Ecuador mittels Sentinel-1 SAR-Daten, um einen kosteneffizienten Ansatz für das Monitoring
  in dieser wolkenreichen Region zu entwickeln. Mithilfe der freien Software Sentinel Application
  Platform (SNAP) wurde eine Zeitreihenanalyse durchgeführt und deren Ergebnisse im Hinblick
  auf Störungsökologie diskutiert. Die Ergebnisse deuten darauf hin, dass Hangrutschungen zur
  Biodiversität der Region beitragen.%
}%


%%%%%%%%%%%%%%%%%%%%%%%%%%%%%%%%%%%%%%%%%%%

% time: Friday 13:45
% URL: https://pretalx.com/fossgis2023/talk/fossgis2026-83912-gottingen-auf-mehr-als-einen-blick-erfahrungen-aus-der-einfuhrung-von-masterportal/

%
\newTimeslot{13:45}
\noindent\abstractHSzwei{%
  Olaf Willenbrock, Olaf Schimmich%
}{%
  Göttingen auf mehr als einen Blick~-- Erfahrungen aus der Einführung von Masterportal%
}{%
}{%
  Die Stadt Göttingen ersetzt in Zusammenarbeit mit der grit GmbH ihr derzeitiges Geoportal und
  wechselt auf die Open Source Anwendung Masterportal. Dabei wird das Masterportal aus einem
  Container eingesetzt. Der Vortrag zeigt, wie die Bereitstellung, Konfiguration und der Betrieb
  über Docker und Orchestrierungsskripte umgesetzt wurden~-- praxisnah und übertragbar für andere
  Kommunen.%
}%


%%%%%%%%%%%%%%%%%%%%%%%%%%%%%%%%%%%%%%%%%%%

% time: Friday 13:45
% URL: https://pretalx.com/fossgis2023/talk/fossgis2026-83608-osm-indoor-routing/

%

\noindent\abstractHSdrei{%
  Volker Krause%
}{%
  OSM Indoor Routing%
}{%
}{%
  Routing innerhalb von Gebäuden ist beispielsweise für die korrekte Ermittlung von Umstiegen im
  Bahnverkehr relevant, umso mehr wenn verschiedene Aspekte der Barrierefreiheit berücksichtigt
  werden sollen. Dieser Vortrag stellt eine Umsetzung von flächenbasiertem Indoor-Routing auf OSM
  Daten und mittels ausschließlich freier Software vor.%
}%


%%%%%%%%%%%%%%%%%%%%%%%%%%%%%%%%%%%%%%%%%%%

% time: Friday 14:20
% URL: https://pretalx.com/fossgis2023/talk/fossgis2026-84161-wie-die-community-zusammenhalt-erfahrungen-aus-der-ip-masterportal/

%
\newTimeslot{14:20}
\noindent\abstractHSzwei{%
  Maren Michaelis%
}{%
  Wie die Community zusammenhält:  Erfahrungen aus der IP Masterportal%
}{%
}{%
  Das Masterportal ist ein Open Source Geoviewer, dessen Weiterentwicklung von über 55 öffentlichen
  Institutionen in der Implementierungspartnerschaft (IP) gesteuert wird. Damit diese effizient
  zusammenarbeiten können, ist ein professionelles Community Management unverzichtbar. Der Vortrag
  veranschaulicht anhand praktischer Beispiele, welche Möglichkeiten die Vielfalt der Community
  bietet, welche Herausforderungen das Community Management bewältigt und welche Erfahrungen daraus
  resultieren.%
}%


%%%%%%%%%%%%%%%%%%%%%%%%%%%%%%%%%%%%%%%%%%%

% time: Friday 14:20
% URL: https://pretalx.com/fossgis2023/talk/fossgis2026-83879-erwartete-qualitatssicherung/

%

\noindent\abstractHSdrei{%
  Dr. Roland Olbricht%
}{%
  Erwartete Qualitätssicherung%
}{%
}{%
  Die OpenStreetMap-Regeln für Daten sind fürs Mappen optimiert, mit Hinblick auf den Erfolg auch
  zurecht. Von Datennutzern wird erwartet, dass sie die Daten aufbereiten. Es gibt jedoch keinen
  konkreten und umfassenden Regelsatz. Der Vortrag versucht, Erwartungen an die Aufbereitung zu
  sammeln und zu systematisieren.%
}%


%%%%%%%%%%%%%%%%%%%%%%%%%%%%%%%%%%%%%%%%%%%

% time: Friday 14:20
% URL: https://pretalx.com/fossgis2023/talk/fossgis2026-83204-ki-basierte-kartierung-geschutzter-pflanzengesellschaften-aus-fernerkundungsbildern/

%

\noindent\abstractHSvier{%
  Dr. Christopher Frank, Kröber, Dr. Alexander Willner%
}{%
  KI-basierte Kartierung geschützter Pflanzengesellschaften aus Fernerkundungsbildern%
}{%
}{%
  KIBI nutzt offene Geodaten und frei verfügbare KI-Modelle, um geschützte Pflanzengesellschaften
  automatisiert aus Fernerkundungsdaten zu kartieren. Ziel ist es, aktuelle, landesweite
  Grünlandinformationen zu gewinnen, die Infrastrukturplanung und  das Umweltmonitoring zu
  erleichtern. Das Projekt entwickelt basierend auf offenen Modellen und stellt Ergebnisse für
  Rheinland-Pfalz als Open Data bereit.%
}%


%%%%%%%%%%%%%%%%%%%%%%%%%%%%%%%%%%%%%%%%%%%

% time: Friday 14:55
% URL: https://pretalx.com/fossgis2023/talk/fossgis2026-84134-wms-time-nutzung-von-diensten-mit-zeitlicher-komponente/

%
\newTimeslot{14:55}
\noindent\abstractHSeins{%
  Thekla Wirkus%
}{%
  WMS-Time: Nutzung von Diensten mit zeitlicher Komponente%
}{%
}{%
  WMS-Time unterstützt Anfragen mit Zeitangabe; dies erfolgt durch Bereitstellung eines
  TIME-Parameters mit einem Zeitwert. MapServer und QGIS unterstützen die Verwendung dieses
  TIME-Parameters.%
}%


%%%%%%%%%%%%%%%%%%%%%%%%%%%%%%%%%%%%%%%%%%%

% time: Friday 14:55
% URL: https://pretalx.com/fossgis2023/talk/fossgis2026-82321-wo-sind-meine-ways-geblieben/

%

\noindent\abstractHSdrei{%
  Michael Reichert%
}{%
  Wo sind meine Ways geblieben?%
}{%
}{%
  Dieser Vortrag stellt *Where are my ways?* vor, eine Toolchain und Webanwendung zum Auffinden
  gelöschter OpenStreetMap-Ways. Schwerpunkt des Vortrags sind die Herausforderungen, die sich aus
  dem OpenStreetMap-Datenmodell ergeben und welche Lücken für Vandalen offen bleiben.%
}%


%%%%%%%%%%%%%%%%%%%%%%%%%%%%%%%%%%%%%%%%%%%

% time: Friday 14:55
% URL: https://pretalx.com/fossgis2023/talk/fossgis2026-84307-muschelbankerkennung-ki-basierte-luftbildanalyse-fur-das-monitoring-des-wattenmeers/

%

\noindent\abstractHSvier{%
  Florian  Eiben, Uwe Breitkopf%
}{%
  Muschelbankerkennung~-- KI-basierte Luftbildanalyse für das Monitoring des Wattenmeers%
}{%
}{%
  Muschelbänke sind ein zentraler Bestandteil des Ökosystems im Niedersächsischen Wattenmeer. Um
  ihre Bestände zu überwachen, werden jährlich Luftbilder ausgewertet. Zur Unterstützung dieses
  Monitorings hat das Landesamt für Geoinformation und Landesvermessung Niedersachsen (LGLN) ein
  KI-System entwickelt.%
}%


%%%%%%%%%%%%%%%%%%%%%%%%%%%%%%%%%%%%%%%%%%%

% time: Friday 15:00
% URL: https://pretalx.com/fossgis2023/talk/fossgis2026-83044-geo-vadis-kartographie-und-geoinformation/

%
\newTimeslot{15:00}
\noindent\abstractHSeins{%
  Florian Stender%
}{%
  Geo vadis, Kartographie und Geoinformation?%
}{%
}{%
  Dieser kurze Vortrag soll einen Denkanstoß zu den aktuellen Entwicklungen und gegenläufigen Trends
  rund um Geodaten geben. Einerseits ist unsere Gesellschaft mit einer größer werdenden Menge an
  räumlichen Daten und einer wachsenden Vielfalt von Anwendungsbereichen konfrontiert. Andererseits
  wird es immer schwerer, das Handwerk der raumbezogenen Informatik sowie der fachgerechten
  graphischen Veredelung professionell zu erlernen.
  Zählt hier nur noch Quantität vor Qualität?%
}%


%%%%%%%%%%%%%%%%%%%%%%%%%%%%%%%%%%%%%%%%%%%

% time: Friday 15:30
% URL: https://pretalx.com/fossgis2023/talk/fossgis2026-85174-abschlussveranstaltung/

%
\newTimeslot{15:30}
\noindent\abstractHSeins{%
  %
}{%
  Abschlussveranstaltung%
}{%
}{%
  Drei spannende Konferenztage gehen zu Ende. Ein gemeinsamer Abschluss soll erfolgen mit Rückblick
  auf die Konferenz und das Erlebte. Natürlich auch mit einem Ausblick auf kommende Veranstaltungen
  und die Konfernz im Jahr 2027.%
}%


%%%%%%%%%%%%%%%%%%%%%%%%%%%%%%%%%%%%%%%%%%%

% time: Friday 16:00
% URL: https://pretalx.com/fossgis2023/talk/fossgis2026-85175-sektempfang-im-foyer/

%
\newTimeslot{16:00}
\noindent\abstractHSeins{%
  %
}{%
  Sektempfang im Foyer%
}{%
}{%
  Der FOSSGIS e.V. lädt alle Mitglieder des FOSSGIS-Vereins, Freunde und Interessierte herzlich zum
  Sektempfang zum Ausklang der FOSSGIS 2026 am FOSSGIS-Vereins-Stand ein.%
}%


%%%%%%%%%%%%%%%%%%%%%%%%%%%%%%%%%%%%%%%%%%%
